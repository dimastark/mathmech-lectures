\documentclass[oneside]{book}

\usepackage{amsfonts}
\usepackage[mathletters]{ucs}
\usepackage[utf8x]{inputenc}
\usepackage[russian]{babel}

\setlength{\parskip}{1px}

\begin{document}

\begin{titlepage}
	\author{Верников Борис Муневич}
    \title{Алгебра и Геометрия}
    \maketitle
\end{titlepage}

\tableofcontents

\chapter{Введение в Алгебру}
\section{Множества и отображения}

\begin{flushleft}
	\begin{tabular}{ l l l l }
	    $\mathbb{N}$ \textit{- натуральные} & $\mathbb{Z}$ \textit{- целые} & $\mathbb{Q}$ \textit{- рациональные} & $\mathbb{R}$ \textit{- действительные} \\
        $A \cup B$ \textit{- объединение} & $A \cap B$ \textit{- пересечение} & $A \setminus B$ \textit{- разность} & $\overline{A}$ \textit{- дополнение} \\
        \\
        $A \subseteq B$ & \multicolumn{3}{l}{ \textit{A подмножество B, B - надмножество A} } \\
        $x \in A$ & \multicolumn{3}{ l }{\textit{x принадлежит A, A - содержит x}} \\
        \\
        $A = B \iff A \subseteq B $ \textit{и} $ A \subseteq B$ \\
    \end{tabular}
\end{flushleft}

\end{document}
